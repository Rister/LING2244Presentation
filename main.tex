\documentclass{beamer}
\usepackage[utf8]{inputenc}

% Themes
\usetheme{Malmoe}
\usecolortheme{seahorse}

% Extra macro setup - can remove if unwanted
% This puts a section slide at the beginning of each section
% Copied from Stackexchange at https://tex.stackexchange.com/a/178803
\AtBeginSection[]{ 
	\begin{frame}
	\vfill
	\centering
	\begin{beamercolorbox}[sep=8pt,center,shadow=true,rounded=true]{title}
		\usebeamerfont{title}\secname\par%
	\end{beamercolorbox}
	\vfill
	\end{frame}
}

\makeatletter
\patchcmd{\beamer@section}{{#2}{\the\c@page}}{{#1}{\the\c@page}}{}{}
\makeatother

\title{Mothers as Singing Mentors for Infants}
\subtitle{Article Authors: Gudmundsdottir, H.R. and Trehub, S.E.}
\author{Juhani Dickinson \\ Jeremy Rist}


\begin{document}

\frame{\titlepage}

\section{Introduction}

\subsection{Authors}
\begin{frame}
	Sandra E. Trehub PhD , Professor Emeritus with the University of Toronto\pause

	Helga Rut Gudmundsdottir PhD, Associate Professor of Music Education at the University of Iceland
\end{frame}

\subsection*{}
\begin{frame}
	An investigation into:
	\begin{itemize}
		\item The impact of singing on infant development and emotional state
			\pause
		\item The role of mothers as a mentor for children both through the use of music and to teach music
			\pause
		\item Infants' perception of music and ability to detect changes in musical properties
	\end{itemize}
\end{frame}

\section{Objectives}
\begin{frame}
	\begin{itemize}
		\item Review the literature surrounding the maternal activity of singing to a child
			\pause
		\item Observe and catalog the emotional response of a child to maternal song
	\end{itemize}
\end{frame}

\section{Research Questions}
\begin{frame}
	\begin{enumerate}
		\item What kinds of songs are usually sung to infants?
			\pause
		\item When are infants usually sung to?
			\pause
		\item What mannerisms or methods are used when singing to infants, and how does this differ from non-infant-directed singing?
			\pause
		\item How do infants respond to singing?
			\pause
		\item To what extent can infants detect changes in musical properties?
			\pause
		\item What stages do children go through in learning to sing?
			\pause
		\item Are all mothers singing mentors?
	\end{enumerate}
\end{frame}

\section{Literature Review}
\begin{frame}
	% Contrast studies of the informal singing of adults with the studies investigating maternal singing
	\begin{itemize}
		\item Studies of informal singing \\ 
			(Dalla Bella et al. 2007; Pfrodresher et al. 2010; Welch 1985) 
			\pause
		\item Studies of Maternal Singing and its consequences \\ 
			(Rock et al. 1999; Trehub et al. 1997) \\ 
			(Masataka 1999; Trainor 1996; Trainor et al. 1997) 
			\pause
		\item Visual gestures \\ 
			(Longhi 1999) 
			\pause
		\item Infant and toddler reproduction \\ 
			(Gergely and Csibra 2006; Over and Carpenter 2013) 
	\end{itemize}
\end{frame}


\section{Results and Discussion}
\subsection*{Kinds of Songs}
\begin{frame}
	\frametitle{What kinds of songs are usually sung to infants?}
	\pause
	It depends on the culture:\\
	\hfill\newline
	Developing nations: lullabies.\\
	Developed nations: play songs, occasionally lullabies.
\end{frame}

\subsection*{When are Infants Sung To}

\begin{frame}
	\frametitle{When are infants usually sung to?}
	\begin{itemize}
		\item Stressful situations
		\item Bedtime
		\item Play
	\end{itemize}
\end{frame}

\subsection*{Singing to Infants}

\begin{frame}
	\frametitle{What mannerisms or methods are used when singing to infants?}
	\begin{itemize}
		\item Smiling
		\item Exagerated expression
		\item Singing slowly
		\item Higher Pitch
	\end{itemize}
\end{frame}

\subsection*{Infant Response}
\begin{frame}
	\frametitle{How do infants respond to singing?}
	\begin{itemize}
		\item Attentiveness: infants listen longer to maternal singing than non-maternal singing.
		\item Singing modulates stress arousal (alertness).
		\pause
		\item Calms down distressed infant.
	\end{itemize}
\end{frame}

\subsection*{Ability to Detect Musical Changes}

\begin{frame}
	\frametitle{To what extent can infants detect changes in musical properties?}
	\begin{itemize}
		\item Infants can detect a small pitch change in a single note of melody.
		\item Sometimes they can detect interval changes when the melodic contour stays the same.
		\item Infants can detect changes in rhythmic grouping even when tempo changes are happening.
		\item Infants more accurately detected timing changes in songs with duple meter than with triple meter.
		\item With ambiguous meter: movement affects what infants recognise as familiar.
	\end{itemize}
\end{frame}

\begin{frame}
	\frametitle{To what extent can infants detect changes in musical properties?}
	\begin{itemize}
		\item Young infants ``learn'' to detect changes in their culture's music similarly to how they learn to detect sounds in their native language.
		\item Adults can readily detect small pitch changes in songs based on a familiar major scale, but not on an unfamiliar scale.
		\pause
		\item 9-month-old infants detect small pitch changes in both song types.
		\item 6-month-old infants detected changes in meter in both simple-metered and complex-metered Balkan music; adults only in simple-metered music.
	\end{itemize}
\end{frame}

\subsection*{Stages of Learning to Sing}

\begin{frame}
	\frametitle{What stages do children go through in learning to sing?}
	\begin{itemize}
		\item Repeated syllables
		\item Pitch contour with copied rhythm
		\item Singing made-up songs
	\end{itemize}
\end{frame}

\subsection*{All Mothers as Mentors}
\begin{frame}
	\frametitle{Are all mothers singing mentors?}
	\begin{itemize}
		\item Middle-class mothers typically sing to infants even if not inclined to sing in other contexts.\\
		\item Some mothers sing to infants infrequently, if at all.\\
		\item Limited exposure to maternal singing may reflect limited caregiving in other areas.\\
		\item \pause %Ask question here
		\item Deaf signing mothers provide other sensitive one-on-one interaction.\\
		\item Signed communication with infants has slower tempo, more exaggerated movements.\\
		\item Infants of deaf signing parents receive rhythmic dance-like gestures instead of rhythmic singing.\\
	\end{itemize}
\end{frame}

\section{Conclusion}
\begin{frame}
	\begin{enumerate}
		\item In developing countries, primarily lullabies; in developed countries, primarily play songs and occasionally lullabies.
		\item Primarily when at play, secondarily during routines (diaper changing, feeding, etc.).
		\item Often smiling, gives a "happier" sound/conveys happiness; increased usage of gestures; %--TODO: finish this one--
		\item Infant stress is modulated, happy infants become happier, distressed infants become happy.
		\item Infants are quite capable in recognising changes in musical properties, and young infants are capable of detecting changes adults have grown out of recognising.
		\item %<>
		\item Yes, in a sense. While not always using vocal interaction (deaf family), sign language and other gestures towards the child are more rythmic than when interacting with adults.
	\end{enumerate}
\end{frame}

\section{Further Reading}
\begin{frame}
	Musical lives of infants (published in The Oxford Handbook of Music Education, Vol. 1) -- M. Adachi and S.E. Trehub, 2012\\
	The Oxford Handbook of Children's Musical Cultures -- P.H. Campbell and T. Wiggins, 2013\\
	Mothers and Others: The Evolutionary Origins of Mutual Understanding -- S.B. Hrdy, 2009\\
	The developmental origins of musicality (publised in Nature Neuroscience 6) -- S.E. Trehub, 2003
\end{frame}

\end{document}
