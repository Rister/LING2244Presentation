\documentclass{beamer}
%Themes
%\usetheme{Malmoe}
\usecolortheme{beetle}
\usepackage[utf8]{inputenc} %Is this needed?

\title{Mothers as Singing Mentors for Infants}
\subtitle{Article Authors: Gudmundsdottir, H.R. and Trehub, S.E.}
\author[Juhani Dickinson, Jeremy Rist]{Juhani Dickinson \\ Jeremy Rist}

%Extra macro setup - can remove if unwanted
%This puts a section slide at the beginning of each section
\AtBeginSection[]{ %Copied from Stackexchange at https://tex.stackexchange.com/a/178803
	\begin{frame}
	\vfill
	\centering
	\begin{beamercolorbox}[sep=8pt,center,shadow=true,rounded=true]{title}
		\usebeamerfont{title}\secname\par%
	\end{beamercolorbox}
	\vfill
	\end{frame}
}
\makeatletter
\patchcmd{\beamer@section}{{#2}{\the\c@page}}{{#1}{\the\c@page}}{}{}
\makeatother


\begin{document}

\begin{frame}
	\titlepage
\end{frame}

%\begin{frame}
%	\frametitle{Outline}
%	\tableofcontents
%\end{frame}

\section{Introduction}
\begin{frame}
	An investigation into:
	\begin{itemize}
	\item The impact of singing on infant development and emotional state
	\item The role of mothers as a mentor for children both through the use of music and to teach music
	\item Infants' perception of music and ability to detect changes in musical properties
	\end{itemize}
\end{frame}

\section{Objectives}
\begin{frame}
	%<>
\end{frame}

\section{Research Questions}
\begin{frame}
	\begin{enumerate}
		\item What kinds of songs are usually sung to infants?
		\item When are infants usually sung to?
		\item What mannerisms or methods are used when singing to infants, and how does this differ from non-infant-directed singing?
		\item How do infants respond to singing?
		\item To what extent can infants detect changes in musical properties?
		\item What stages do children go through in learning to sing?
		\item Are all mothers singing mentors?
	\end{enumerate}
\end{frame}

\section{Results and Discussion}
\subsection*{Kinds of Songs}
\begin{frame}
	\frametitle{What kinds of songs are usually sung to infants?}
	%Ask question here
\end{frame}
\begin{frame}
	\frametitle{What kinds of songs are usually sung to infants?}
	It depends on the culture.\\
	Developing nations: lullabies.\\
	Developed nations: play songs, occasionally lullabies.
\end{frame}

\subsection*{When are Infants Sung To}
\begin{frame}
	\frametitle{When are infants usually sung to?}
	%<>
\end{frame}

\subsection*{Singing to Infants}
\begin{frame}
	\frametitle{What mannerisms or methods are used when singing to infants?}
	%<>
\end{frame}

\subsection*{Infant Response}
\begin{frame}
	\frametitle{How do infants respond to singing?}
	Attentiveness: infants listen longer to maternal singing than non-maternal singing.\\
	Singing modulates stress arousal (alertness).
	%As question here
\end{frame}

\begin{frame}
	\frametitle{How do infants respond to singing?}
	Attentiveness: infants listen longer to maternal singing than non-maternal singing.\\
	Singing modulates stress arousal (alertness).\\
	Calms down distressed infant.
\end{frame}

\subsection*{Ability to Detect Musical Changes}
\begin{frame}
	\frametitle{To what extent can infants detect changes in musical properties?}
	Infants can detect a small pitch change in a single note of melody.
	Sometimes they can detect interval changes when the melodic contour stays the same.
	Infants can detect changes in rhythmic grouping even when tempo changes are happening.
	Infants more accurately detected timing changes in songs with duple meter than with triple meter.
	With ambiuous meter: movement affects what infants recognise as familiar.
\end{frame}

\begin{frame}
	\frametitle{To what extent can infants detect changes in musical properties?}
	Young infants ``learn'' to detect changes in their culture's music similarly to how they learn to detect sounds in their native language.
	Adults can readily detect small pitch changes in songs based on a familiar major scale, but not on an unfamiliar scale.
	%Ask question here
\end{frame}

\begin{frame}
	\frametitle{To what extent can infants detect changes in musical properties?}
	Young infants ``learn'' to detect changes in their culture's music similarly to how they learn to detect sounds in their native language.
	Adults can readily detect small pitch changes in songs based on a familiar major scale, but not on an unfamiliar scale.
	9-month-old infants detect small pitch changes in both song types.
	6-month-old infants detected changes in meter in both simple-metered and complex-metered Balkan music; adults only in simple-metered music.
\end{frame}

\subsection*{Stages of Learning to Sing}
\begin{frame}
	\frametitle{What stages do children go through in learning to sing?}
	%<>
\end{frame}

\subsection*{All Mothers as Mentors}
\begin{frame}
	\frametitle{Are all mothers singing mentors?}
	%<>
\end{frame}

\section{Conclusion}
\begin{frame}
	\item %<>
	\item Primarily when at play, secondarily during routines (diaper changing, feeding, etc.).
	\item Often smiling, gives a "happier" sound/conveys happiness; increased usage of gestures; %--TODO: finish this one--
	\item Infant stress is modulated, happy infants become happier, distressed infants become happy.
	\item %<>
	\item %<>
	\item Yes, in a sense. While not always using vocal interaction (deaf family), sign language and other gestures towards the child are more rythmic than when interacting with adults.
\end{frame}

\section{Further Reading}
\begin{frame}
	Musical lives of infants (published in The Oxford Handbook of Music Education, Vol. 1) -- M. Adachi and S.E. Trehub, 2012\\
	The Oxford Handbook of Children's Musical Cultures -- P.H. Campbell and T. Wiggins, 2013\\
	Mothers and Others: The Evolutionary Origins of Mutual Understanding -- S.B. Hrdy, 2009\\
	The developmental origins of musicality (publised in Nature Neuroscience 6) -- S.E. Trehub, 2003
	
\end{frame}

\end{document}